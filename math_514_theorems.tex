%% 
%% This is file, `math_514_theorems.tex',
%% generated with the extract package.
%% 
%% Generated on :  2018/11/18,19:47
%% From source  :  "Math 514 Lecture Notes".tex
%% Using options:  active,generate=Math_514_Theorems,extract-env={theorem,corollary,claim}
%% 
\documentclass[12pt]{article}
  \usepackage{amsthm}
  \usepackage{amsfonts, amsmath}
  \usepackage[dvipsnames]{xcolor}
  %\newtheorem{theorem}{Theorem}
  \theoremstyle{definition}
  \definecolor{Tm}{rgb}{0,0,0.80}
  %\newtheorem{definition}{Definition}
  \newtheorem{definition}{\color{NavyBlue}{\textbf{Definition}}}
  \newtheorem{theorem}{\color{ForestGreen}{\textbf{Theorem}}}
  \newtheorem{corollary}{Corollary}
  %\newtheorem{example}{Example}
   \usepackage{bm}
\newcommand{\e}{\epsilon}
%\newcommand{\d}{\delta}
\newcommand{\D}{\Delta}
\newcommand{\R}{\mathbb{R}}
\newcommand{\Z}{\mathbb{Z}}
\usepackage{mathtools}
\DeclarePairedDelimiterX{\inp}[2]{\langle}{\rangle}{#1, #2}
\newcommand{\norm}[1]{\lVert#1\rVert}
\newcommand{\x}{\bm{x}}
\newcommand{\xib}{\bm{\xi}}
\usepackage{mathrsfs}
\usepackage{algpseudocode,algorithm,algorithmicx}
\algnewcommand\algorithmicforeach{\textbf{for each}}
\algdef{S}[FOR]{ForEach}[1]{\algorithmicforeach\ #1\ \algorithmicdo}
\newcommand*\Let[2]{\State #1 $\gets$ #2}

\begin{document}

\begin{theorem}(The Mean Value Theorem)
Suppose $f$ is a real-valued function, defined and continuous on the closed interval $[a,b] \in \R$ and $f$ differentiable on the open interval $(a,b)$. Then there exists a number $\xi \in (a,b)$ such that
\begin{equation}
f(b) - f(a) = f'(\xi) (b - a)
\end{equation}
\end{theorem}

\begin{theorem}[Taylor's Theorem]
Suppose that $n$ is a nonnegative integer, and $f$ is a real-valued function, defined and continuous on the closed interval $[a,b]$ of $\R$, such that the derivatives of $f$ of order up to and including $n$ are defined and continuous on the closed interval $[a,b]$. Suppose further that $f^{(n)}$ is differentiable on the open interval $(a,b)$. Then, for each value of $x \in [a,b]$, there exists a number $\xi = \xi(x)$ in the open interval $(a,b)$ such that
\begin{equation}
f(x) = f(a) + (x-a)f'(a) + \cdots + \frac{(x-a)^n}{n!} f^{(n)}(a) + \frac{(x-a)^{n+1}}{(n+1)!} f^{(n+1)}(\xi)
\end{equation}
\end{theorem}

\begin{theorem}(Existence of Root)\label{zeroexists}
Let $f$ be a real-valued function, defined and continuous on a bounded closed interval $[a,b]$ of the real line. Assume further, that $f(a)f(b) \leq 0$; then, there exists $\xi$ in $[a,b]$ such that $f(\xi) = 0$.
\end{theorem}

\begin{theorem}[Brouwer's Fixed Point Theorem]
Suppose that $g$ is a real-valued function, defined and continuous on a bounded closed interval $[a,b]$ of the real line, and let $g(x) \in [a,b]$ for all $x \in [a,b]$. Then, there exists $\xi \in [a,b]$ such that $\xi = g(\xi)$. $\xi$ is called a fixed point of the function $g$.
\end{theorem}

\begin{theorem}[Contraction Mapping Theorem]
Suppose that $g$ is a real-valued function, defined and continuous on a bounded closed interval $[a,b]$ of the real line, and let $g(x) \in [a,b]$ for all $x \in [a,b]$. Suppose $g$ is a contraction on $[a,b]$. Then, $g$ has a unique fixed point $\xi$ in the interval $[a,b]$. Moreover, the sequence $(x_k)$ defined by simple iteration converges to $\xi$ as $k \to \infty$ for any starting value $x_0$ in $[a,b]$.

Let $\epsilon > 0$ be a certain tolerance, and let $k_0 (\epsilon)$ denote the smallest positive integer such that $x_k$ is no more than $\epsilon$ away from the fixed point $\xi$ (i.e. $|x_k - \xi| \leq \epsilon$) for all $k \geq k_0 (\epsilon)$. Then,
\begin{equation}
k_0 (\epsilon) \leq \left\lfloor \frac{\ln|x_1 - x_0| - \ln(\epsilon (1 - L))}{\ln (1 / L)}\right\rfloor + 1
\end{equation}
\end{theorem}

\begin{theorem}[Contraction Mapping Theorem when Differentiable]
Suppose that $g$ is a real-valued function, defined and continuous on a bounded closed interval $[a,b]$ of the real line, and let $g(x) \in [a,b]$ for all $x \in [a,b]$. Let $\xi= g(\xi) \in [a,b]$ be a fixed point of $g$ (the existence of this point is guaranteed by Brouwer's fixed point theorem). Assume $g$ has a continuous derivative in some neighborhood of $\xi$ with $|g'(\xi)| < 1$. Then the sequence $(x_k)$ defined by simple iteration $x_{k+1} = g(x_k)$, $k \geq 0$, converges to $\xi$ as $k \to \infty$, provided that $x_0$ is close to $\xi$.
\end{theorem}

\begin{theorem}[Unstable Fixed Points]
Suppose that $\xi = g(\xi)$, where the function $g$ has a continuous derivative in some neighborhood of $\xi$, and let $|g'(\xi)| > 1$ (thus $\xi$ is an unstable fixed point). Then the sequence $(x_k)$ defined by simple iteration $x_{k+1} = g(x_k)$, $k \geq 0$, does not converge to $\xi$ from any starting value $x_0$, $x_0 \neq \xi$.
\end{theorem}

\begin{theorem}[Convergence of Newton's Method]
Suppose that $f$ is a continuous real-valued function with continuous second derivative $f''$ defined on the closed interval $I_\delta = [\xi - \delta, \xi + \delta]$, $\delta > 0$, such that $f(\xi) = 0$ and $f''(\xi) \neq 0$. Additionally suppose that there exists a positive constant $A$ such that
\begin{equation}
\frac{|f''(x)|}{|f'(y)|} \leq A \quad \forall x,y, \in I_\delta
\end{equation}
If initially
\begin{equation}
|\xi - x_0| \leq h = \min(\delta, \frac{1}{A})
\end{equation}
then the sequence $(x_k)$ defined by Newton's method converges quadratically to $\xi$.
\end{theorem}

\begin{theorem}[Convergence of Secant Method]
Suppose that $f$ is a real-valued function, defined and continuously differentiable on an interval $I = [\xi - h, \xi + h]$, $h > 0$, with center point $\xi$. Suppose further that $f(\xi) = 0$, $f'(\xi) \neq 0$. Then, the sequence $(x_k)$ defined by the secant method converges at least linearly to $\xi$ provided that $x_0$ and $x_1$ are sufficiently close to $\xi$.
\end{theorem}

\begin{theorem}[The $1$-norm of a matrix is the largest absolute-value column sum]
Let $A \in \R^{m \times n }$ and denote the columns of $A$ by $a_j$, $j=1, \ldots, n$. Then $\norm{A}_1 = \max_{j=1, \ldots, n} \sum_{i=1}^{m} |a_{ij}| =  \max_{j=1, \ldots, n} \norm{a_j}$.
\end{theorem}

\begin{theorem}[The $\infty$-norm of a matrix is the largest absolute-value row sum]
Let $A \in \R^{m \times n }$ and denote the rows of $A$ by $b_i$, $i=1, \ldots, m$. Then $\norm{A}_{\infty} = \max_{i=1, \ldots, m} \sum_{j=1}^{n} |a_{ij}| =  \max_{i=1, \ldots, m} \norm{b_i}$.
\end{theorem}

\begin{theorem}[The $2$-norm of a symmetric positive definite matrix is the maximum absolute value of its eigenvalues]
Let $A$ be a positive definite $n \times n$ matrix. Then
\begin{equation}
\norm{A}_2 = \max_{i=1,\ldots, n} |\lambda_i|
\end{equation}
\end{theorem}

\begin{theorem}(The $2$-norm of a matrix $A_{m \times n}$ equals its largest singular value)
Let $A$ be an $m \times n$ matrix and denote the eigenvalues of the matrix $B = A^TA$ by $\lambda_i$, $i=1,\ldots, n$. Then
\begin{equation}
\norm{A}_2 = \max_i \sqrt{\lambda_i}
\end{equation}
The square roots of the (nonnegative) eigenvalues of $A^TA$ are referred to as the singular values of $A$.
\end{theorem}

\begin{theorem}[Properties of spd matrices]
Let $A$ be an $n \times n$ real, spd matrix. Then
\begin{enumerate}
\item $a_{ii} > 0$ for all $i=1, \ldots, n$ (the diagonal elements of $A$ are positive).
\item $A x_i \ = \lambda_i x_i \implies \lambda_i \in \R_{>0}, \x \in \R^n \setminus \{0\}$ (the eigenvalues of $A$ are real and positive, and the eigenvectors of $A$ belong to $\R^n \setminus \{0\}$).
\item $x_i \perp x_j$ if $\lambda_i \neq \lambda_j$ (the eigenvectors of distinct eigenvalues of $A$ are orthogonal)
\item $\det(A) > 0$ (the determinant of $A$ is positive)
\item Every submatrix $B$ of $A$ obtained by deleting any set of rows and the corresponding set of columns from $A$ is symmetric and positive definite (in particular, every principal submatrix is positive definite).
\end{enumerate}
\end{theorem}

\begin{theorem}
If $A$ is spd, then there exists a lower diagonal matrix $L$ such that $A = L L^T$. This is called the Cholesky decomposition.
\end{theorem}

\begin{theorem}[Contraction Mapping Theorem in $\R^n$]
Suppose $D$ is a closed subset of $\R^n$ and $g: \R^n \to \R^n$ is defined on $D$, and $g(D) \subset D$. Suppose further that $g$ is a contraction on $D$ in the $\infty$-norm. Then,
\begin{enumerate}
\item $g$ has a unique fixed point $\xib \in D$
\item The sequence $(\x^{(k)})$ defined by $\x^{(k+1)} = g(\x^{k})$ converges to $\xib$ for any starting value $x^{(0)} \in D$.
\end{enumerate}
\end{theorem}

\begin{theorem}
Let $g = (g_1, \ldots, g_n)^T : \R^n \to \R^n$ be a function defined and continuous on a closed set $D \subset \R^n$. Let $\xib \in D$ be a fixed point of $g$. Suppose the first partial derivatives of each $g_i$ are defined and continuous in some (open) neighborhood $N(\xib) \in D$ of $\xib$, with
\begin{equation}
\norm{J_g(\xib)}_\infty < 1
\end{equation}
Then there exists $\e > 0$ such that $g (\bar B_\e(\xib)) \subset \bar B_\e(\xib)$, and the sequence $\x^{(k+1)} = g(\x^{k})$ converges to $\xib$ for all $\x^{(0)} \in \bar B_\e(\xib)$ (in other words, the sequence converges to $\xib$ as long as $\x^{(0)}$ is close enough to $\xib$).
\end{theorem}

\begin{theorem}
Suppose $f(\xib) = 0$, that in some (open) neighborhood $N(\xib)$ of $\xib$, where $f$ is defined and continuous, all the second-order partial derivatives of $f$ are defined and continuous, and that the Jacobian matrix $J_f(\x^{(k)})$ of $f$ at the point $\xib$ is nonsingular. Then the sequence defined by Newton's method converges to $\xib$ provided that $\x^{(0)}$ is sufficiently close to $\xib$.
\end{theorem}

\begin{theorem}(Abel(-Ruffini) Theorem, or ``No-go Theorem'')
There is no algebraic solution (that is, a solution expressed in terms of radicals) to general polynomial equations of degree five or higher with arbitrary coefficients.
\end{theorem}

\begin{theorem}[Convergence of Power Iteration]
Suppose $|\lambda_1|  > |\lambda_2| \geq \ldots \geq |\lambda_n|$ and $q_1^T v^{(0)} \neq 0$. Then the iterates of power iteration satisfy
\begin{align}
\norm{v^{(k)} - (\pm q_1)} &= \mathcal{O}\left( \left\vert \frac{\lambda_2}{\lambda_1} \right\vert^k \right) \tag{error of eigenvector} \\
|\lambda^{(k)} - \lambda_1 | &= \mathcal{O}\left( \left\vert \frac{\lambda_2}{\lambda_1} \right\vert^{2k} \right) \tag{error of eigenvalue}
\end{align}
\end{theorem}

\begin{theorem}[Error of Rayleigh Quotient]
Let $x_1$ be the eigenvector that corresponds to the largest (in absolute value) eigenvalue. If $\norm{x - x_1} = \mathcal{O}(\e)$, then
\begin{equation}
\left\vert \frac{\inp{x}{Ax}}{\inp{x}{x}}  - \lambda_1 \right\vert = \mathcal{O}(\e^2)
\end{equation}
\end{theorem}

\begin{theorem}(Equivalence of Simultaneous Iteration and the QR Algorithm)
Simultaneous Iteration and the QR Algorithm generate identical sequences of matrices $\underline R^{(k)}, \underline Q^{(k)}, A^{(k)}$. Both give
\begin{align}
(a): A^{(k)} &= \underline Q^{(k)} \underline R^{(k)} \tag{$QR$ factorization of the $k$th power of $A$}\\
(b): A^{(k)} &= (\underline Q^{(k)})^T A \underline Q^{(k)} \tag{projection}
\end{align}
\end{theorem}

\begin{theorem}[Error of Lagrange interpolation polynomial]
Suppose that $n\geq 0$ and the $f$ is a real-valued function, defined and continuous on the closed real interval $[a,b]$, such that derivative of $f$ or order $n+1$ exists and is continuous on $[a,b]$. Then, with $x \in [a,b]$, there exists $\xi = \xi(x)$ in $(a,b)$ such that
\begin{equation}
f(x) - p(x) = \frac{f^{(n+1)}(\xi)}{n!} \prod_{k=0}^n (x-x_k)
\end{equation}
\end{theorem}

\begin{theorem}[Chebyshev grid]

\end{theorem}

\begin{theorem}[OP Recurrence Relation]
A set of orthogonal polynomials $\{\phi\}_{i=0}^\infty$ satisfies
\begin{equation}
\phi_{n+1} = (\alpha_n x + \beta_n)\phi_n + \gamma_n \phi_{n-1}
\end{equation}
\end{theorem}

\begin{theorem}
If $\{\phi\}_{i=0}^\infty$, then $\phi_n(x)$ has $n$ real roots, called Gaussian quadratures.
\end{theorem}

\begin{theorem}[Locations of Gaussian Quadratures from Recurrence Relation]
Give the recurrence relation
\begin{equation}
\phi_{n+1} = (\alpha_n x + \beta_n)\phi_n + \gamma_n \phi_{n-1}
\end{equation}
we can rewrite this as
\begin{equation}
\alpha_n x \phi_n = \phi_{n+1} - \beta_n \phi_n - \gamma_n \phi_{n-1}
\end{equation}
Thus for constants $a_n,b_n,c_n$ we have that
\begin{equation}
x \phi_n = \phi_{n-1} +  b_n \phi_n + c_n \phi_{n+1}
\end{equation}
where this equality holds for all $x$ in the domain. We can write this system in matrix form as follows
\begin{equation}
x
\begin{pmatrix}
\phi_0(x) \\ \phi_1(x) \\ \vdots \\ \vdots \\ \phi_n(x)
\end{pmatrix}
=
\begin{pmatrix}
b_0 & c_0 \\
a_1 & b_1 & c_1 \\
& a_2 & b_2 & \ddots \\
& & \ddots & \ddots \\
& & & & & c_{n-1} \\
& & & & a_n & b_n
\end{pmatrix}
\begin{pmatrix}
\phi_0(x) \\ \phi_1(x) \\ \vdots \\ \vdots \\ \phi_n(x)
\end{pmatrix}
+
\begin{pmatrix}
0 \\ 0 \\ \vdots \\ \vdots \\ c_n \phi_{n+1}
\end{pmatrix}
\end{equation}
where $A$ is the matrix of coefficients. We want to find the roots $\phi_{n+1}(x_i) = 0$, where $i=1,\ldots,n+1$. Then the eigenvalues of $A$ are the zeros of $\phi_{n+1}$. In sum
\begin{equation}
\text{GQ of } \phi_{n+1} = eig(A)
\end{equation}
\end{theorem}

\begin{theorem}
Suppose. $f(x) \in \mathbb{P}_{2N+1}$. Then
\begin{equation}
\int_{a}^b f(x)w(x)dx = \sum_{i=0}^N f(x_k) w_k
\end{equation}
if $\{x_0, \ldots, x_N\}$ are the GQ (roots) of $\phi_{N+1}$, where
\begin{equation}
w_k = \int_a^b l_k(x)w(x)dx
\end{equation}
where $l_k(x)$ is a Lagrange polynomial.
\end{theorem}

\begin{theorem}
Let $f(x) \in \mathbb{P}_{N+1}$. Then
\begin{equation}
\alpha_i = \inp{f}{\phi_i} = \int_a^b f(x)\phi_i(x)w(x)dx = \sum_{k=0}^N f(x_k)\phi_i(x_k)w_k = c_i
\end{equation}
That is the projection coefficients $c_i$ are equal to the numerical representation $\alpha_i$, where the grid points are the GQ of $\phi_{N+1}$.
\end{theorem}

\begin{theorem}
We interpolate $f$ as follows:
\begin{equation}
p(x) = \sum_{n=0}^N c_n \phi_n(x)
\end{equation}
such that $p(x_i) = f(x_i)$ where the $x_i$ are the GQ of $\phi_{N+1}$. Then
\begin{equation}
\begin{bmatrix}
\phi_0(x_0) & \phi_1(x_0) & \hdots & \phi_N(x_0) \\
\phi_0(x_1) & \phi_1(x_1) & \hdots & \phi_N(x_1) \\
\vdots & \vdots & & \vdots  \\
\phi_0(x_N) & \phi_1(x_N) & \hdots & \phi_N(x_N)
\end{bmatrix}
\begin{bmatrix}
c_0 \\ \vdots \\ \vdots \\ c_N
\end{bmatrix}
=
\begin{bmatrix}
f(x_0) \\ f(x_1) \\ \vdots \\ \vdots  \\ f(x_N)
\end{bmatrix}
\end{equation}
Then $A$, the matrix above, is almost unitary. In particular,
\begin{equation}
A^T \cdot W \cdot A = I
\end{equation}
where $W$ is a diagonal matrix with elements $w_0, w_1, \ldots, w_N$.
\end{theorem}

\begin{theorem}[Projection the best approximation in the $L^2$-norm:]
$p_N(x)$ is the best approximation in the $L^2$-norm:
\begin{equation}
\norm{f-p_N(x)}^2_2 \leq \norm{f-q(x)}^2_2
\end{equation}
for all $q \in \mathbb{P_N}$.
\end{theorem}

\begin{theorem}
If $f \in \mathbb{P}_{2N+1}$
\begin{equation}
\int f(x)w(x)dx = \sum_{i=0}^N f(x_i)w_i
\end{equation}
\end{theorem}

\begin{theorem}[First mean value theorem for definite integrals]
If $f:[a,b] \to \R$ is continuous and $g$ is an integrable function that does not change sign on $[a,b]$, then there exists $c \in [a,b]$ such that
\begin{equation}
\int_a^b f(x)g(x) dx = f(c)\int_a^b g(x)dx
\end{equation}
\end{theorem}

\end{document}
